\documentclass[a4paper,11pt]{article} \usepackage{anysize}
\marginsize{2cm}{2cm}{1cm}{1cm}
%\textwidth 6.0in \textheight = 664pt
\usepackage{xltxtra}
\usepackage{xunicode}
\usepackage{graphicx}
\usepackage{color}
\usepackage[usenames,dvipsnames]{xcolor}
\usepackage{xgreek}
\usepackage{fancyvrb}
\usepackage{minted}
\usepackage{listings}
\usepackage{enumitem} \usepackage{framed} \usepackage{relsize}
\usepackage{float} \setmainfont[Mapping=TeX-text]{FreeSerif}
\begin{document}

\renewcommand{\theenumi}{\roman{enumi}}
\include{title/title}

\section*{sys\_curse: σχεδιασμός υποδομής σημείωσης διεργασιών στον πυρήνα Linux}
\subsection*{σχεδιασμός υποδομής curse}
Κάθε κατάρα θα έχει
\begin{itemize}
    \item ποιες ήταν οι σημαντικότερες αποφάσεις σχεδιασμού που πήρατε και 

        ??

    \item τι επεμβάσεις στον κώδικα ή τις δομές δεδομένων του πυρήνα προδιαγράψατε και γιατί.

        Θα τροποποιηθεί το task\_struct ώστε να περιέχει την πληροφορία για
        τις κατάρες καθώς και όσες κλήσεις ή κομμάτια του συστήματος θέλουμε
        να γνωρίζουν και να αναλαμβάνουν κάποια δράση ανάλογα με την κατάρα
        που έχει η διεργασία.

    \item πώς εξασφαλίσατε την κληρονομικότητα των κατάρων.

        Για την υλοποίηση του μηχανισμού των κατάρων θα προσθέσουμε στο
        process control block ένα ακόμη πεδίο. Αυτό θα είναι ένα uint32\_t το
        οποίο θα χρησιμοποιείται ως bitfield. Εκεί θα "γράφουμε" τις κατάρες
        για κάθε διεργασία. Το PCB είναι κατάλληλη θέση για την υλοποίηση που
        χρειαζόμαστε καθώς είναι μοναδικό για κάθε διεργασία, και
        κληρονομείται ανάμεσα στα forks με πολιτική COW. Έτσι εξασφαλίζουμε
        τόσο την κληρονομικότητα στις κατάρες όσο και τη μικρή επιβάρυνση του
        συστήματος.

    \item τι προβλέπει ο σχεδιασμός σας για το ενδεχόμενο ταυτόχρονης πρόσβασης πολλών διεργασιών.

        Ο μηχανισμός θα συγχρονίζεται με χρήση κάποιου mutex.

    \item με ποιο τρόπο ένας προγραμματιστής μπορεί να υλοποιήσει μια κατάρα με την υποδομή σας.

        Για να υλοποιήσει μια κατάρα, θα πρέπει να αυξήσει τον αριθμό τους
        (σε κάποιο \#define N\_CURSES), να ορίσει τη συνάρτηση που θα χειρίζεται
        τη συγκεκριμένη κατάρα, και να την προσθέσει στο curses\_struct που θα
        κρατάει. Τότε κάθε κομμάτι του πυρήνα που καλεί την curse\_action θα
        μπορεί να χειριστεί και αυτή την κατάρα. 

    \item πώς εξασφαλίσατε ότι το ίδιο εκτελέσιμο αρχείο μπορεί να λειτουργήσει σε διαφορετικά συστήματα με διαφορετικές κατάρες.

        Το εκτελέσιμο απλά ζητάει από τον πυρήνα να θέσει στη διεργασία x την
        κατάρα y. Το εκτελέσιμο δε χρειάζεται να γνωρίζει κάτι άλλο για τη
        συγκεκριμένη κατάρα, ξαναχτίζοντας το με την αντίστοιχη βιβλιοθήκη (αν
        είναι στατική) ή και όχι (αν η βιβλιοθήκη είναι δυναμική) το
        εκτελέσιμο θα ζητήσει από τη βιβλιοθήκη να καταραστεί μια διεργασία με
        τη συγκεκριμένη κατάρα καλώντας την αντίστοιχη συνάρτηση. Έτσι η
        υλοποίηση κρύβεται από το χρήστη και το ίδιο εκτελέσιμο μπορεί να
        χρησιμοποιηθεί ανεξάρτητα από το ποιες κατάρες είναι υλοποιημένες.

    \item ποια εκτιμάτε ότι θα είναι η επιβάρυνση των παρεμβάσεών σας στο λειτουρ- γικό σύστημα;

        Η επιβάρυνση στο σύστημα θα είναι ελάχιστη. Ο μηχανισμός των κατάρων
        προσθέτει ένα μόνο ακέραιο στο task\_struct που όσο το σύστημα
        δε χρησιμοποιείται δεν αντιγράφεται καν (COW). Απεναντίας δίνοντας μας
        τη δυνατότητα να τροποποιήσουμε το πώς μια διεργασία συμπεριφέρεται
        (βλέπε caching) μας επιτρέπει να επιταχύνουμε το σύστημα. Από την άλλη
        πλευρά η υλοποίηση των κατάρων πρέπει να γίνει προσεκτικά ώστε να μην
        τρέχουμε κώδικα που δε χρειάζεται λαμβάνοντας υπόψιν πως στο μεγαλύερο
        ποσοστό η τιμή που θα έχει η κατάρα θα είναι μηδενική.

\end{itemize}
\subsection*{σχεδιασμός κατάρας no-fs-cache}

\begin{itemize}

    \item τι επεμβάσεις στον κώδικα ή τις δομές δεδομένων του πυρήνα προδιαγράψατε και γιατί.

        Η κατάρα no-fs-cache για να υλοποιηθεί είναι απαραίτητο να
        τροποποιήσουμε την κλήση συστήματος open. Συγκεκριμένα θα πρέπει να
        καλεί την curse\_action ώστε να λαμβάνουμε υπόψιν την κατάρα.

    \item σε ποιες κλήσεις συστήματος προδιαγράφει κλήσεις fadvise (ενότητα 3.3) ο σχεδιασμός σας και πώς.

        Η fadvise θα καλείται κάθε φορά που επιθυμούμε να κάνουμε open ένα
        αρχείο δηλώνοντας στον μηχανισμό του cache πως τα δεδομένα θα
        διαβαστούν μόνο μία φορά (POSIX\_FADV\_NOREUSE). Έτσι δεν έχει νόημα να γίνει το caching
        εξ'αρχής. Ακόμη, αν μια διεργασία έχει ανοιχτά αρχεία και σε αυτά θα
        δηλώσουμε πως δεν είναι απαραίτητο το caching (POSIX\_FADV\_DONTNEED).

    \item ποια εκτιμάτε ότι θα είναι η αξιοπιστία σε ασφάλεια και σταθερότητα του συστήματος που σχεδιάσατε.

        Το σύστημα αυτό θα λειτουργεί σύμφωνα με τις προδιαγραφές που δώθηκαν.

    \item ποια εκτιμάτε ότι θα είναι η επιβάρυνση του λειτουργικού συστήματος από τη λειτουργία της κατάρας.

        Επιβάρυνση πιθανώς να έχουμε στην περίπτωση που έχουμε πολλά αρχεία
        ανοιχτά και πρέπει να απελευθερώσουμε πολλές σελίδες ωστόσο η
        επιτάχυνση που θα έχουμε από μη χρησιμοποίηση του cache mechanism όταν
        αυτός δεν είναι απαραίτητος μάλλον θα επιταχύνει το σύστημα.

    \item ποια είναι η εκτίμησή σας για την αποτελεσματικότητα του σχεδιασμού σας.

        ??

\end{itemize}
\end{document}
